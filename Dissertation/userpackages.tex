\usepackage{fr-longtable}    %ради \endlasthead

% Листинги с исходным кодом программ
\usepackage{fancyvrb}
\usepackage{listings}
\lccode`\~=0\relax %Без этого хака из-за особенностей пакета listings перестают работать конструкции с \MakeLowercase и т. п. в (xe|lua)latex

% Русская традиция начертания греческих букв
\usepackage{upgreek} % прямые греческие ради русской традиции

%%% Микротипографика
%\ifnumequal{\value{draft}}{0}{% Только если у нас режим чистовика
%    \usepackage[final, babel, shrink=45]{microtype}[2016/05/14] % улучшает представление букв и слов в строках, может помочь при наличии отдельно висящих слов
%}{}

% Отметка о версии черновика на каждой странице
% Чтобы работало надо в своей локальной копии по инструкции
% https://www.ctan.org/pkg/gitinfo2 создать небходимые файлы в папке
% ./git/hooks
% If you’re familiar with tweaking git, you can probably work it out for
% yourself. If not, I suggest you follow these steps:
% 1. First, you need a git repository and working tree. For this example,
% let’s suppose that the root of the working tree is in ~/compsci
% 2. Copy the file post-xxx-sample.txt (which is in the same folder of
% your TEX distribution as this pdf) into the git hooks directory in your
% working copy. In our example case, you should end up with a file called
% ~/compsci/.git/hooks/post-checkout
% 3. If you’re using a unix-like system, don’t forget to make the file executable.
% Just how you do this is outside the scope of this manual, but one
% possible way is with commands such as this:
% chmod g+x post-checkout.
% 4. Test your setup with “git checkout master” (or another suitable branch
% name). This should generate copies of gitHeadInfo.gin in the directories
% you intended.
% 5. Now make two more copies of this file in the same directory (hooks),
% calling them post-commit and post-merge, and you’re done. As before,
% users of unix-like systems should ensure these files are marked as
% executable.
\ifnumequal{\value{draft}}{1}{% Черновик
   \IfFileExists{.git/gitHeadInfo.gin}{
      \usepackage[mark,pcount]{gitinfo2}
      \renewcommand{\gitMark}{rev.\gitAbbrevHash\quad\gitCommitterEmail\quad\gitAuthorIsoDate}
      \renewcommand{\gitMarkFormat}{\rmfamily\color{Gray}\small\bfseries}
   }{}
}{}

% Мои пакеты
\usepackage{pdfpages}

\lstdefinelanguage{JavaScript}{
  keywords={break, case, catch, continue, debugger, default, delete, do, else, false, finally, for, function, if, in, instanceof, new, null, return, switch, this, throw, true, try, typeof, var, const, void, while, with},
  morecomment=[l]{//},
  morecomment=[s]{/*}{*/},
  morestring=[b]',
  morestring=[b]",
  ndkeywords={class, export, boolean, throw, implements, import, this},
  keywordstyle=\bfseries,
  ndkeywordstyle=\bfseries,
  identifierstyle=\color{black},
  commentstyle=\ttfamily,
  stringstyle=\ttfamily,
  sensitive=true
}

\lstdefinelanguage{Scheme}{
   keywords={define, define-syntax, define-macro, lambda, define-stream, stream-lambda, begin, call-with-current-continuation, call/cc,
   call-with-input-file, call-with-output-file, case, cond,
   do, else, for-each, if,
   let*, let, let-syntax, letrec, letrec-syntax,
   let-values, let*-values,
   and, or, not, delay, force,
   quasiquote, quote, unquote, unquote-splicing,
   map, fold, syntax, syntax-rules, eval, environment, query},
   morecomment=[l]{;;},
   morecomment=[s]{/*}{*/},
   morestring=[b]',
   morestring=[b]",
   ndkeywords={class, export, boolean, throw, implements, import, this},
   keywordstyle=\bfseries,
   ndkeywordstyle=\bfseries,
   identifierstyle=\color{black},
   commentstyle=\ttfamily,
   stringstyle=\ttfamily,
   sensitive=true
}

\lstdefinelanguage{llvm}{
  morecomment = [l]{;},
  morestring=[b]", 
  sensitive = true,
  classoffset=0,
  morekeywords={
    define, declare, global, constant,
    internal, external, private,
    linkonce, linkonce_odr, weak, weak_odr, appending,
    common, extern_weak,
    thread_local, dllimport, dllexport,
    hidden, protected, default,
    except, deplibs,
    volatile, fastcc, coldcc, cc, ccc,
    x86_stdcallcc, x86_fastcallcc,
    ptx_kernel, ptx_device,
    signext, zeroext, inreg, sret, nounwind, noreturn,
    nocapture, byval, nest, readnone, readonly, noalias, uwtable,
    inlinehint, noinline, alwaysinline, optsize, ssp, sspreq,
    noredzone, noimplicitfloat, naked, alignstack,
    module, asm, align, tail, to,
    addrspace, section, alias, sideeffect, c, gc,
    target, datalayout, triple,
    blockaddress
  },
  classoffset=1, keywordstyle=\color{purple},
  morekeywords={
    fadd, sub, fsub, mul, fmul,
    sdiv, udiv, fdiv, srem, urem, frem,
    and, or, xor,
    icmp, fcmp,
    eq, ne, ugt, uge, ult, ule, sgt, sge, slt, sle,
    oeq, ogt, oge, olt, ole, one, ord, ueq, ugt, uge,
    ult, ule, une, uno,
    nuw, nsw, exact, inbounds,
    phi, call, select, shl, lshr, ashr, va_arg,
    trunc, zext, sext,
    fptrunc, fpext, fptoui, fptosi, uitofp, sitofp,
    ptrtoint, inttoptr, bitcast,
    ret, br, indirectbr, switch, invoke, unwind, unreachable,
    malloc, alloca, free, load, store, getelementptr,
    extractelement, insertelement, shufflevector,
    extractvalue, insertvalue,
  },
  alsoletter={\%},
  keywordsprefix={\%},
}