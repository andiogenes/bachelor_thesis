\chapter*{Введение}                         % Заголовок
\addcontentsline{toc}{chapter}{Введение}    % Добавляем его в оглавление

\newcommand{\actuality}{}
\newcommand{\progress}{}
\newcommand{\aim}{{\textbf\aimTXT}}
\newcommand{\tasks}{\textbf{\tasksTXT}}
\newcommand{\novelty}{\textbf{\noveltyTXT}}
\newcommand{\influence}{\textbf{\influenceTXT}}
\newcommand{\methods}{\textbf{\methodsTXT}}
\newcommand{\defpositions}{\textbf{\defpositionsTXT}}
\newcommand{\reliability}{\textbf{\reliabilityTXT}}
\newcommand{\probation}{\textbf{\probationTXT}}
\newcommand{\contribution}{\textbf{\contributionTXT}}
\newcommand{\publications}{\textbf{\publicationsTXT}}


{\actuality} В последние несколько лет в сфере информационных технологий
набирает популярность технология No-code / Low-code -- разработка без
использования программного кода или с малым его количеством. Компании,
связывающие себя с этим трендом, предоставляют прикладным программистам
и людям без опыта в программировании специальные платформы, позволяющие
создавать прикладное программное обеспечение с помощью графического
пользовательского интерфейса и настройки вместо программирования.

Как замечают сторонники такого подхода к разработке программного обеспечения,
в части программных продуктов написание программного кода занимает
малую долю их жизненного цикла в сравнении с другими этапами, такими как
развертывание, поддержка, масштабирование, обеспечение безопасности и 
интегрирование с другими продуктами; поэтому имеется интерес в разработке
и распространении инструментов разработки, позволяющих свести написание кода
к минимуму и ориентированных в первую очередь на непрограммистов.

Программными системами, которые можно улучшить путём интеграции с инструментами
разработки без использования программного кода, являются сервисы для 
проектирования пользовательских интерфейсов и прототипирования. Продукты
подобного плана предлагают развитые инструменты для макетирования, однако в
построении интерактивных прототипов их возможности сильно ограничены - большая
часть таких систем позволяет настраивать только переходы между макетами. 
При таком подходе игнорируется самая важная часть функционального прототипа 
программы -- модель данных приложения. Без представления о движении данных
в программе невозможно понять ожидаемую механизм взаимодействия пользователя
и системы и построить полноценный функциональный прототип приложения.

Часть данной работы посвящена проектированию и разработке системы, позволяющей
решить проблему построения пользовательского взаимодействия, учитывающего
моделирование данных,  в программных пакетах разработки интерфейсов и
прототипирования приложений. В качестве решения предлагается 
визуальный язык, реализующий концепции функционального программирования.
С использованием такого языка, пользователь системы строит поток данных,
вычислительные узлы которого -- функции. Функции строго типизированы, то
есть возможны только соединения входов и выходов, типы которых совпадают.

Для разделения данных, пользовательского интерфейса и управляющей логики
предлагается использование функционального шаблона проектирования
Model-View-Update (MVU, "Модель-Представление-Обновление"), оформляющего
программный цикл в формирование образа данных в модуле "Модель", обновление
данных по запросу в модуле "Обновление" и отображение в виде интерфейса 
пользователю в модуле "Представление". Часть данной работы посвящена
проектированию и разработке интегрированной среды для визуального языка, 
реализующей графический пользовательский интерфейс для работы с 
компонентами шаблона MVU.

Особенностью визуальных языков программирования является их высокоуровневость
и ориентированность на людей, не имеющих опыта с программированием, поэтому
среднестатистический пользователь такого языка может написать неэффективную
по производительности программу, и даже не знать об этом. В свою очередь,
особенностью функциональных языков программирования, включая язык,
рассматриваемый в рамках данной работы, является отсутствие управляющих
конструкций для организации повторяющихся вычислений -- циклов, меток перехода.
Вместо них в функциональных языках используется рекурсия. Как известно,
неосторожное использование рекурсии может привести к неожиданным последствиям --
например, переполнению стека вызовов программы и ее аварийному завершению.
О методах устранения рекурсии, проектировании и разработке отпимизирующего 
компилятора для визуального языка, поддерживающего оптимизацию нескольких
часто встречающихся шаблонов использования рекурсии, также пойдет речь в 
данной работе.
 % Характеристика работы по структуре во введении и в автореферате не отличается (ГОСТ Р 7.0.11, пункты 5.3.1 и 9.2.1), потому её загружаем из одного и того же внешнего файла, предварительно задав форму выделения некоторым параметрам

%\textbf{Объем и структура работы.} Работа состоит из~введения,
%\formbytotal{totalchapter}{глав}{ы}{}{},
%заключения и
%\formbytotal{totalappendix}{приложен}{ия}{ий}{}.
%% на случай ошибок оставляю исходный кусок на месте, закомментированным
%Полный объём диссертации составляет  \ref*{TotPages}~страницу
%с~\totalfigures{}~рисунками и~\totaltables{}~таблицами. Список литературы
%содержит \total{citenum}~наименований.
%
%Полный объём работы составляет
%\formbytotal{TotPages}{страниц}{у}{ы}{}, включая
%\formbytotal{totalcount@figure}{рисун}{ок}{ка}{ков} и
%\formbytotal{totalcount@table}{таблиц}{у}{ы}{}.
%Список литературы содержит
%\formbytotal{citenum}{наименован}{ие}{ия}{ий}.
