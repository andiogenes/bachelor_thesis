\chapter*{\centerline{Заключение}}                       % Заголовок
\addcontentsline{toc}{chapter}{Заключение}  % Добавляем его в оглавление

%% Согласно ГОСТ Р 7.0.11-2011:
%% 5.3.3 В заключении диссертации излагают итоги выполненного исследования, рекомендации, перспективы дальнейшей разработки темы.
%% 9.2.3 В заключении автореферата диссертации излагают итоги данного исследования, рекомендации и перспективы дальнейшей разработки темы.
%% Поэтому имеет смысл сделать эту часть общей и загрузить из одного файла в автореферат и в диссертацию:

В результате самостоятельной работы студента в семестре и преддипломной практики были решены следующие задачи:

\begin{enumerate}
    \item Рассмотрены технологии, алгоритмы, методы предметной области, формализованы в отдельной главе пояснительной записки выпускной квалификационной работы.
    \item Разработан проект визуального языка программирования, поддерживающего функциональную парадигму.
    \item Построена клиент-серверная архитектура среды разработки на спроектированном языке с поддержкой компиляции программы в целевой код на языке JavaScript.
    \item Спроектирован интерфейс интегрированной среды программирования, разработан функциональный прототип среды с использованием технологии HTML5.
    \item Формализованы методы оптимизации рекурсии в программах на спроектированном языке программирования.
\end{enumerate}

В ближайшее время планируется:

\begin{enumerate}
    \item Интегрировать среду программирования и компилятор в единую программную систему.
    \item Провести полномасштабное тестирование функционального прототипа системы.
    \item Зафиксировать соответствующую информацию в пояснительной записке к выпускной квалификационной работе студента.
\end{enumerate}
