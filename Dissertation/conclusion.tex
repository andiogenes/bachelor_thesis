\chapter*{\centerline{Заключение}}                       % Заголовок
\addcontentsline{toc}{chapter}{Заключение}  % Добавляем его в оглавление

%% Согласно ГОСТ Р 7.0.11-2011:
%% 5.3.3 В заключении диссертации излагают итоги выполненного исследования, рекомендации, перспективы дальнейшей разработки темы.
%% 9.2.3 В заключении автореферата диссертации излагают итоги данного исследования, рекомендации и перспективы дальнейшей разработки темы.
%% Поэтому имеет смысл сделать эту часть общей и загрузить из одного файла в автореферат и в диссертацию:

В процессе работы была подробна исследована предметная область задачи.
Были рассмотрены основные идеи визуального программирования, изучены
основания и концепции функционального программирования; рассмотрен
современный подход к построению компиляторов, в том числе, виды промежуточных
представлений программы, методы оптимизации рекурсивных программ компиляторами.
Был проведен обзор популярных визуальных средств программирования и 
прототипирования приложений.

На основе изученных концепций был спроектирован визуальный
язык функционального программирования, основанный на понятии
чистых функций, и поддерживающий такие элементы функциональной парадигмы,
как функции высшего порядка и отложенные вычисления.

Для разработки программ на спроектированном языке была предложена концепция
интегрированной среды программирования, основанной на архитектурном шаблоне
построения интерактивных функциональных приложений.

Результатом работы стала система визуального функционального программирования,
включающая:

\begin{itemize}
    \item среду программирования на спроектированном языке;
    \item оптимизирующий транслятор в JavaScript-код, поддерживающий 
    такие оптимизации, как устранение хвостовой рекурсии,
    мемоизация вычислений рекурсивных функций.
\end{itemize}

Среда и транслятор связаны между собой отношением <<клиент-сервер>> в
одноименной архитектуре.

В будущем планируется продолжить исследование данной темы с целью:

\begin{itemize}
    \item выявления других способов оптимизации рекурсии компиляторами;
    \item введения в язык строгой семантики типов, подобной имеющимся в
    языках семейства ML.
\end{itemize}

Тезисы о проектировании системы опубликованы в материалах Всероссийской
научно-технической конференции <<Наука и молодежь -- 2021>>.