\chapter{Проектирование программного обеспечения}\label{ch:ch2}

\section{Проектирование визуального языка программирования}\label{sec:ch2/sec1}

Языку дано название \textbf{Flovver}. Flovver --- аппликативный язык, то есть, вычислительный процесс в нем основывается на вычислении 
результата функции от заданного числа аргументов и передачи этого значения в другие функции.

На уровне семантики в языке существует всего-лишь один тип объектов --- функция.

Функция переводит от 1 до N значений заданных входящих типов в одно значение выходящего типа.

В Flovver функции представлены диаграммами вида, представленного на рисунке \ref{fig:functions}.

\begin{figure}[ht]
	\centering
	\includegraphics [scale=1.0] {functions}
	\caption{Представление функции в Flovver}
	\label{fig:functions}
\end{figure}
\FloatBarrier

Здесь блок $f$ --- некоторая функция типа $input 1$ $\rightarrow$ ... $\rightarrow$ $input n$ $\rightarrow$ $output$. 
Левая часть блока --- входы функции, правая часть блока --- выход функции.

Из правой части выходит дуга, которую можно соединить с входом другой функции (рисунок \ref{fig:functions_composition}).

\begin{figure}[ht]
	\centering
	\includegraphics [scale=0.7] {functions_composition}
	\caption{Композиция функций в Flovver}
	\label{fig:functions_composition}
\end{figure}
\FloatBarrier

Если на вход переданы не все аргументы, считается, что функция недоопределена, и выходную дугу создать нельзя.

Функцию и переданные аргументы можно частично применить, проведя дугу из нижней стороны блока к месту использования (рисунок \ref{fig:thunks}).

\begin{figure}[ht]
	\centering
	\includegraphics [scale=1.0] {thunks}
	\caption{Синтаксис частичного применения функции}
	\label{fig:thunks}
\end{figure}
\FloatBarrier

В результате получим из данной функцию от (нестрого) меньшего числа аргументов, ранее переданные аргументы будут зафиксированы (рисунок \ref{fig:thunks_example}).

\begin{figure}[ht]
	\centering
	\includegraphics [scale=0.6] {thunks_example}
	\caption{Пример частичного применения функции}
	\label{fig:thunks_example}
\end{figure}
\FloatBarrier

Чтобы вычислить функцию, передаваемую как значение, можно использовать специальную функцию $apply$ (рисунок \ref{fig:apply_thunk}).

\begin{figure}[ht]
	\centering
	\includegraphics [scale=0.8] {apply_thunk}
	\caption{Пример использования специальной функции apply}
	\label{fig:apply_thunk}
\end{figure}
\FloatBarrier

Функция $apply$ принимает первым параметром функцию от N аргументов, следующие 2 ... N+1 параметров - параметры, подставляемые в первую функцию (рисунок \ref{fig:apply}).

\begin{figure}[ht]
	\centering
	\includegraphics [scale=1.0] {apply}
	\caption{Синтаксис функции apply}
	\label{fig:apply}
\end{figure}
\FloatBarrier

Построение новых функций из заданных представлено на рисунке \ref{fig:function_body}.

\begin{figure}[ht]
	\centering
	\includegraphics [scale=0.9] {function_body}
	\caption{Построение новой функции}
	\label{fig:function_body}
\end{figure}
\FloatBarrier

Здесь блок f --- тело функции, левая сторона блока --- входы функции, правая сторона блока --- выход функции; внутри блока находится функция $g$, к которой применяются входы $f$, результат $g$ передается на выход $f$.

Для поддержки создания рекурсивных функций внутри тела функции возможно создание специального блока $self$, который является ссылкой на объявляемую функцию (рисунок \ref{fig:function_body_self}).

\begin{figure}[ht]
	\centering
	\includegraphics [scale=0.9] {function_body_self}
	\caption{Определение рекурсивной функции}
	\label{fig:function_body_self}
\end{figure}
\FloatBarrier

Можно считать, что функции с $self$ применены к комбинатору неподвижной точки. Комбинатором неподвижной точки (также Y-комбинатором) называют специальную функцию высшего порядка, которая вычисляет неподвижную точку другой функции. 
Практическая ценность такой функции - возможность использовать рекурсию для анонимных функций без необходимости определения им имени.

\FloatBarrier

%\section{Проектирование общей архитектуры приложения}\label{sec:ch2/sec2}

\section{Проектирование вариантов использования приложения}\label{sec:ch2/sec3}

Были написаны возможные сценарии действий пользователей.

\begin{enumerate}
    \item Тим-лид Иван побывал на стоячей встрече, откуда узнал, что его команда будет разрабатывать приложение “Список задач”. Иван хочет сделать прототип, чтобы объяснить задачу остальным членам команды.

    Он создает новый проект в приложении Flovver, описывает модель данных: текстовый список задач (нужно где-то хранить задачи). 

    Далее Иван начинает делать макет - добавляет поле для ввода, кнопку и текстовый список.

    Иван сразу настроил макет, чтобы текст для списка брался из списка в модели данных.

    Теперь он хочет, чтобы при нажатии на кнопку текст из формы добавлялся в список. Он рисует простую программу на Flovver, которая это и делает.

    Иван включает прототип, смотрит, что все работает как надо, собирает программу и идет показывать другим членам команды.

    \item У программиста Олега ответственная задача - ему надо добавить функцию “Удаление элементов” из списка в приложение “Список задач”. Олег открывает проект в Flovver и видит, что ему надо поменять только макет приложения и логику программы. В макете Олег добавляет элементам списка кнопку “Удалить” и идет менять логику программы. Олег добавляет функцию удаления в программу, запускает прототип, проверяет, что все работает и идет на ревью, показывать другим членам команды.

    \item Дизайнеру Афанасию поручили подобрать цвет кнопки “Добавить задачу” вместе с заказчиком Михаилом. Афанасий открыл проект в Flovver и по желанию Михаила меняет цвет кнопки. Через пятнадцать минут Михаил определился с цветом кнопки и попросил выделить текст в кнопке курсивом, что Афанасий и сделал. Афанасий показал рабочий прототип Михаилу, Михаил пожал Афанасию руку и пошел выписывать чек в отдел работы с клиентами.

    \item Аналитик Игорь решил, что в приложении “Список задач” кроме задачи нужно указывать также ориентировочную дату выполнения. Он открыл проект “Список задач” в Flovver, заменил модель данных “Текстовый список задач” на модель “Список кортежей (Дата, Текст задачи)”. Игорь попытался запустить прототип, но Flovver отказался запускать его, сообщив “В программе не соответствуют типы данных - ожидался Текстовый список задач, получили Список кортежей (Дата, Текст задачи)”. К счастью, Игорь не запаниковал, отменил внесенные изменения и позвал программиста Олега. Олег выслушал Игоря, изменил модель и программу. В программе появилась новая функция, а Игорь в итоге ничего не сломал.
\end{enumerate}

\section{Проектирование интерфейса приложения}\label{sec:ch2/sec4}

\subsection{Проектирование меню и функционально-модульной структуры}\label{sec:ch2/sec4/subsec1}

При открытии программы должен появиться список настроек проекта. В верхней части рабочей области находится навигационное меню с возможностью перехода в модули «Модель», «Обновление», «Представление», «Настройки». Также на верхней панели находятся кнопки «Собрать» и «Запустить», при нажатии на которых происходит сборка проекта, запуск сборки проекта в отдельном окне соответственно.

В модуле «Настройки» содержится форма для управления флагами проекта – «Устранение хвостовой рекурсии»,  «Устранение рекурсии Фибоначчи», «Устранение общей рекурсии», которые представлены элементами управления типа «Флаг» (англ. – check-box). Запись флагов в системное хранилище осуществляется при нажатии на кнопку «Сохранить» в нижней части формы.

В модулях «Модель», «Представление», «Обновление» содержится рабочая область, представляющая пустое пространство, на котором можно размещать элементы предметной области модуля используя механизм «потяни-и-отпусти» (англ. – drag-and-drop). Поверх рабочего окна в модулях находятся две боковые панели – слева – панель с сущностями предметной области модуля, которые можно выбирать наведением и нажатием курсора, добавлять в рабочую область перемещением курсора с зажатой левой кнопкой мыши. Панели можно сворачивать, нажимая на кнопку с соответствующим символом (иконки типа «Шеврон налево», «Шеврон направо»). В правой части содержится панель свойств текущего объекта. При нажатии левой кнопкой мыши на объект рабочей области, для него отображается соответствующий набор свойств – имя объекта, его позиция и т.д. Значения свойств изменяются с использованием соответствующих элементов управления – форм ввода, кнопок. При изменении объекта в рабочей области меняются значения его свойств на панели «Свойства» и наоборот. Компоненты, соответствующие функциональным блокам, представлены в следующем параграфе.

Объекты модуля «Модель» можно назначить моделью проекта. На панели «Свойства» содержится раздел «Роли» с кнопкой «Назначить моделью». При нажатии на нее, выделеный объект помечается как модель. В рабочем пространстве модель визуально представлена объектом с голубым фоном. Объекты типа «Вариант» можно также назначить типом сообщений проекта. При фокусе на варианте в раздел «Роли» добавляется новая кнопка «Назначить типом сообщений». Тип-сообщение выделяется в рабочем пространстве зеленой обводкой объекта.

\subsection{Прототипирование отдельных блоков и компонентов}\label{sec:ch2/sec4/subsec2}

\begin{figure}[ht]
	\centering
	\includegraphics [scale=0.3] {settings_screen}
	\caption{Экран модуля <<Настройки>>, являющийся стартовым}
	\label{fig:settings_screen}
\end{figure}

\begin{figure}[ht]
	\centering
	\includegraphics [scale=0.3] {model_screen}
	\caption{Экран модуля <<Модель>>}
	\label{fig:model_screen}
\end{figure}

\begin{figure}[ht]
	\centering
	\includegraphics [scale=0.3] {view_screen}
	\caption{Экран модуля <<Представление>>}
	\label{fig:view_screen}
\end{figure}

\begin{figure}[ht]
	\centering
	\includegraphics [scale=0.3] {update_screen}
	\caption{Экран модуля <<Обновление>>}
	\label{fig:update_screen}
\end{figure}

\begin{figure}[ht]
	\centering
	\includegraphics [scale=0.7] {dropdown}
	\caption{Компонент <<Выпадающее окно>>, появляющийся при нажатии на кнопку <<Проект>> на навигационной панели}
	\label{fig:dropdown}
\end{figure}

\begin{figure}[ht]
	\centering
	\includegraphics [scale=0.7] {widgets_panel}
	\caption{Левая боковая панель <<Виджеты>> модуля <<Представление>>}
	\label{fig:widgets_panel}
\end{figure}

\begin{figure}[ht]
	\centering
	\includegraphics [scale=0.3] {view_work}
	\caption{Рабочая область модуля <<Представление>>}
	\label{fig:view_work}
\end{figure}

\begin{figure}[ht]
	\centering
	\includegraphics [scale=0.7] {view_properties}
	\caption{Правая боковая панель <<Свойства>> модуля <<Представление>>}
	\label{fig:view_properties}
\end{figure}

\begin{figure}[ht]
	\centering
	\includegraphics [scale=0.7] {model_types}
	\caption{Левая боковая панель <<Типы>> модуля <<Модель>>}
	\label{fig:model_types}
\end{figure}

\begin{figure}[ht]
	\centering
	\includegraphics [scale=0.7] {model_model}
	\caption{Компонент <<Псевдоним примитивного типа>> модуля <<Модель>> в состоянии <<Назначен моделью>>}
	\label{fig:model_model}
\end{figure}

\begin{figure}[ht]
	\centering
	\includegraphics [scale=0.7] {model_primitive}
	\caption{Компонент <<Псевдоним примитивного типа>> модуля <<Модель>> в начальном состоянии}
	\label{fig:model_primitive}
\end{figure}

\begin{figure}[ht]
	\centering
	\includegraphics [scale=0.7] {message_variant}
	\caption{Компонент <<Тип-вариант>> модуля <<Модель>> в состоянии <<Назначен типом сообщений>>}
	\label{fig:message_variant}
\end{figure}

\begin{figure}[ht]
	\centering
	\includegraphics [scale=0.7] {model_variant}
	\caption{Компонент <<Тип-вариант>> в состоянии <<Назначен моделью>>}
	\label{fig:model_variant}
\end{figure}

\begin{figure}[ht]
	\centering
	\includegraphics [scale=0.7] {message_model_variant}
	\caption{Компонент <<Тип-вариант>> в состоянии <<Назначен моделью и типом сообщений>>}
	\label{fig:message_model_variant}
\end{figure}

\begin{figure}[ht]
	\centering
	\includegraphics [scale=0.5] {model_properties}
	\caption{Правая боковая панель <<Свойства>> модуля <<Модель>>, выбран компонент <<Вариант>>}
	\label{fig:model_properties}
\end{figure}

\begin{figure}[ht]
	\centering
	\includegraphics [scale=0.7] {update_call}
	\caption{Компонент <<Вызов функции>> модуля <<Обновление>>}
	\label{fig:update_call}
\end{figure}

\begin{figure}[ht]
	\centering
	\includegraphics [scale=0.7] {update_case}
	\caption{Компонент <<Выбор варианта>> модуля <<Обновление>>}
	\label{fig:update_case}
\end{figure}

\begin{figure}[ht]
	\centering
	\includegraphics [scale=0.7] {update_input}
	\caption{Компонент <<Точка входа>> модуля <<Обновление>>}
	\label{fig:update_input}
\end{figure}

\begin{figure}[ht]
	\centering
	\includegraphics [scale=0.7] {update_output}
	\caption{Компонент <<Точка выхода>> модуля «Обновление»}
	\label{fig:update_output}
\end{figure}

\begin{figure}[ht]
	\centering
	\includegraphics [scale=0.5] {update_def}
	\caption{Компонент <<Определение функции>> модуля <<Обновление>>}
	\label{fig:update_def}
\end{figure}

\begin{figure}[ht]
	\centering
	\includegraphics [scale=0.5] {update_def_band}
	\caption{Компонент <<Определение функции>> в состоянии <<Расстягивание>>}
	\label{fig:update_def_band}
\end{figure}

\FloatBarrier

\subsection{Глоссарий}\label{sec:ch2/sec4/subsec3}

\textbf{Приложение} --– комбинация компьютерных инструкций и данных, позволяющая аппаратному обеспечению вычислительной системы выполнять вычисления или функции управления.

\textbf{Интегрированная среда разработки (ИСР)} –-- это программное приложение, которое предоставляет программистам комплексные возможности для разработки программного обеспечения.

\textbf{Виджет} –-- примитив графического интерфейса пользователя, имеющий стандартный внешний вид и выполняющий стандартные действия.

\textbf{Вкладка} –-- элемент графического интерфейса пользователя, который даёт возможность переключения в одном окне приложения между несколькими открытыми документами или предопределёнными наборами элементов интерфейса, когда их доступно несколько, а на выделенном для них пространстве окна можно показывать только один из них.

\textbf{Панель} --– определенное расположение информации, сгруппированной вместе для представления пользователям в окне или всплывающем окне.

\textbf{Кнопка} --– элемент интерфейса компьютерных программ, является метафорой кнопки в технике и, соответственно, изображается схожей с ней и выполняет аналогичные функции. При нажатии на неё происходит программно связанное с этим нажатием действие либо событие.

\textbf{Метод взаимодействия <<Потяни-и-отпусти>>} (англ. --– drag-and-drop) --- способ оперирования элементами интерфейса в интерфейсах пользователя (как графическим, так и текстовым, где элементы GUI реализованы при помощи псевдографики) при помощи манипулятора «мышь» или сенсорного экрана.

\FloatBarrier

\subsection{Полная функционально-модульная схема приложения}\label{sec:ch2/sec4/subsec4}

\begin{figure}[ht]
	\centering
	\includegraphics [scale=0.5] {full_scheme}
	\caption{Полная схема вкладок приложения}
	\label{fig:full_scheme}
\end{figure}

\FloatBarrier