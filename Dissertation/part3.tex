\chapter{Реализация программного обеспечения}\label{ch:ch3}

Для разработки языкового сервера, компилятора и интегрированной среды
программирование было выбрано сочетание Scala, Jetty и Svelte.

\section{Выбор технологий и средств разработки ПО}\label{sec:ch3/sect1}

\subsection{Scala}\label{sec:ch3/sect1/subsect1}

Scala --- это сильно статически типизированный язык программирования общего назначения, поддерживающий как объектно-ориентированное, так и функциональное программирование. Многие дизайнерские решения Scala направлены на устранение проблем Java.

Исходный код Scala можно скомпилировать в байт-код Java и запустить на виртуальной машине Java (JVM). Scala обеспечивает языковую совместимость с Java, поэтому поддерживает любые библиотеки, написанные на любом языке на платформе JVM. Как и Java, Scala является объектно-ориентированным и использует синтаксис, подобный языку программирования C. 

В отличие от Java, Scala имеет множество особенностей языков функционального программирования, таких как Scheme, Standard ML и Haskell, включая каррирование, неизменяемость данных, ленивые вычисления и сопоставление с образцом. Он также имеет расширенную систему типов, поддерживающую алгебраические типы данных, ковариантность и контравариантность, типы высокого порядка и анонимные типы. Другие функции Scala, отсутствующие в Java, включают перегрузку операторов, необязательные параметры, именованные параметры.

\subsection{Svelte}\label{sec:ch3/sect1/subsect2}

Svelte --- это свободно распространяемый компилятор веб-приложений с открытым исходным кодом. Приложения Svelte не содержат ссылок на фреймворки. Вместо этого создание приложения Svelte генерирует код для управления объектной моделью документа, что может уменьшить размер передаваемых файлов, а также улучшить запуск клиента и производительность во время выполнения. Svelte имеет собственный компилятор для преобразования кода приложения в клиентский JavaScript во время сборки. Он написан на TypeScript. 

% \subsection{TailwindCSS}\label{sec:ch3/sect1/subsect3}

\subsection{Jetty}\label{sec:ch3/sect1/subsect4}

Eclipse Jetty --- это веб-сервер Java и контейнер Java--сервлетов. В то время как веб-серверы обычно связаны с передачей документов людям, Jetty часто используется для связи между машинами, как правило, в рамках более крупных программных сред. Jetty разрабатывается как бесплатный проект с открытым исходным кодом в рамках Eclipse Foundation. 

\subsection{Scalatra}\label{sec:ch3/sect1/subsect5}

Scalatra --- это свободный фреймворк веб-приложений с открытым исходным кодом, написанный на Scala. Это порт фреймворка Sinatra, написанного на Ruby. Scalatra --- альтернатива фреймворкам Lift, Play, и Unfiltered.

Scalatra --- это пример микрофреймворка, фреймворка для разработки веб-программного обеспечения, который пытается быть как можно более минимальным.

%\section{Использование паттернов программирования}\label{sec:ch3/sect2}

%\section{Общая диаграмма классов решения}\label{sec:ch3/sect3}

%\section{Описание методов ПО}\label{sec:ch3/sect4}

